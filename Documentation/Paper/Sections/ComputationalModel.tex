\subsection{Overview}
This thesis involved two major components: simulations of islet instances and an evolutionary algorithm which would score, manipulate, and select superior models with some randomization from those simulations. These components involved differences in requisite hardware resources, data processing, and general purpose and thus were executed and developed separately. In this section we consider the reasoning behind choices made regarding the simulations of islet instances.
\subsection{Framework}
NEURON, a set of language and libraries that facilitate the development of computational models of electrically excitable cells made public by the Yale School of Medicine \cite{hines_neuron_1997}, was selected to create and solve the previously mentioned systems of reaction-diffusion and circuit equations. These systems of equations were represented in nmodl, transpiled to c, and called from within a python script to minimize the space and time complexity of what would have been the most computationally expensive portion of the project  \cite{hines_expanding_2000}. NEURON also presented an array of differential equation solvers with varying stabilities and error bounds; for this implementation, the \emph{Cnexp} solver was selected since it is the recommended solver for Hodgkin-Huxley style models and able to solve stiff systems with up to second-order accuracy.
\par In order to best navigate the volume of data produced by these simulations and the significance of subtleties therein, the executables were written in python and dispatched in the University of Florida's high performance computing environment (Hipergator). This allowed for the storage of 10 TB of data on disk, access to over 128 GB of memory, the use of popular and effective packages for scientific computing (i.e. numpy, scipy, etc.), and the eventual parallelization of the evolutionary algorithm (see the Architecture section for more details).
\subsection{Physiological Properties}
Hodgkin-Huxley style models of electrically excitable cells developed with NEURON determine changes in a cell's membrane potential through calculations done on separate currents corresponding to unique ion species. These currents were calculated using properties of ion channels which were represented in nmodl and added to the cell object provided by the framework. This cell object was then added to the space in which the simulation takes place and the values associated with its mechanisms recorded. 
\par All cells contained sodium and potassium ion channels which were represented through the standard Hogdkin-Huxley style gating variables and experimentally determined rate functions. All cells also contained low and high (L and T type, respectively) voltage gated calcium channels with $\alpha$ and $\delta$ cells incorporating a PQ type (high voltage gated channels commonly associated with purkinje fibers) channel \cite{hashimoto_postsynaptic_2011}. Other currents which were not associated with a single protein channel that were considered in the model include but are not limited to a delayed rectifying potassium current (KDR), an ATP-dependent potassium current (KATP), a G-protein inward rectifying potassium current (GIRK), a GABA current, a background potassium current, and a background sodium current. Leak channels were also included in all cells to represent ion flow through perforations in the lipid membrane. 
\par Hormone secretion was modeled differently in $\beta$ cells than in $\alpha$ and $\delta$ cells. In $\beta$ cells, insulin release was considered only in the context of a cyptoplasmic calcium pool while glucagon and somatostatin release in $\alpha$ and $\delta$ cells were considered in the context of priming and depriming factors that attempted to represent exocytosis on the level of individual granules.

\subsection{Connectivity}
It is widely recognized that the major endocrine cells of the pancreas ($\alpha$, $\beta$, $\delta$) coregulate one another through their characteristic secratogogues. Although considerable regulation of peer endocrine cells can be an effect of downstream metabolic pathways stimulated by a secretory molecule, to maintain consistency and simplicity in this model's implementation, only insulin, glucagon, somatostatin, and GABA were considered as having an effect on other cells. That effect was limited to changing secretion rates of characteristic secratogogues and the conductance of the KATP and  GIRK ion channels in the cells that contained them.
\par A unique aspect of this model is that cells are connected according to their spatial distribution and their interaction governed by secretory agents within the previously mentioned reaction diffusion system. For this spatial distribution to most closely resemble physiological islets (mouse islets have a non-uniform distribution of cell types \cite{steiner_pancreatic_2010}), a loss function was used to determine a dynamic probability of cell type given varying radial distances from the center of the islet. As human islets are more uniformly distributed, a simple random number generator was used to determine cell type and in both cases cells were stacked adjacent to one another within a sphere whose radius to equal to half the width of the cuboidal space which contains it.
