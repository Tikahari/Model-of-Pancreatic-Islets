This model is the first to represent the major cells of pancreatic islets within a full islet model. The use of an evolutionary algorithm to parameterize that model and quantify its accuracy was considered by at least one other paper with a similar scope, but this implementation remains unique among computational models of endocrine cells \cite{gurkiewicz_numerical_2007}. However, to produce meaningful data especially to clinicians and those involved in drug development, this model should be modified according to the conditions under which simulations occur, and the intentions of the researcher. This is especially true regarding modification of the scoring metric and accompanying selection criteria. For models which attempt to capture whole islet function, it is suggested that a multi-objective evolutionary algorithm which accounts for the fidelity of multiple variables (i.e. secretion, gating, etc.) for each cell type to distinct data sets be used in contrast to the single-object algorithm which only considered membrane potential in beta cells used in this thesis. It is worth noting that the focus of this thesis was the development of a framework on which future work will build and develop physiologically significant data. Therefore preliminary results are limited and were deemed not worth focusing on in this document.
\par We hope this model can provide the basis for a new class of computational models of pancreatic islets which view these clusters of endocrine cells as interdependent systems which are best understood through interactions between their spatial, physiological, and environmental profiles.
\clearpage
